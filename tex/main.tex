\documentclass[a4paper,10pt]{article}
\usepackage[utf8]{inputenc}
\usepackage{xcolor,colortbl}
\usepackage{nicefrac}  
\usepackage{amssymb}
\usepackage{array}
\definecolor{Gray}{gray}{0.9}
\newcommand{\col}[1]{\multicolumn{2}{c|}{#1}}
%\newcolumntype{gl}{>{\columncolor{Gray}}l}

\newcommand{\calK}{\mathcal{K}}

\begin{document}

\section{Volume Attack using Machine Learning}
Let $D$ be a database of the form $\{(r_1, k_1), \ldots, (r_n, k_n)\}$, where 
search keys $\{k_1, \ldots, k_n\}$ are elements of domain $\calK$. 
Assume that we know the values of all keys in $D$ except the value of key $k_i$. 
Note that, it is not necessary to know the values of $\{r_1, \ldots, r_n\}$. 
The goal is to train a classifier in order to predict the value of $k_i$. 

The training data for the classifier is prepared as follows:
for every $k_j$ in the key domain $\calK$, 
we assume that $k_i = k_j$ (i.e. we set the value of $k_i$ into $k_j$) 
then we perform $N$ uniform queries on the resulting 
database. For each query we count the number of the output records so that 
we will have at the end a histogram, which constitute with the label $k_j$ 
a training record. Note that for each label $k_j$, $m$ histograms may be generated. 
Repeating the process for every $k_j \in \calK$ will result in $m.|\calK|$ training records, 
each consists of $n+1$ features: the number of output records ranges from 0 to $n$. 
Note that for large $n$ one may bucketize the features. 

\paragraph{Experiment 1}
We consider 13 databases. The databases are selected from UCI Adult dataset\footnote{https://archive.ics.uci.edu/ml/datasets/adult}. 
Each database composed of $n = 10^3$ records, but they have key domains with different sizes. 
The considered key is the ``age'' attribute. For domain $calK$ we consider the interval 
from the minimum age to the maximum age in the database. 
We train a classifier for every $k_i \in D$ while considering $N=10^3$ queries for 
each label. That is for each $i$ we assume that we know all other keys but not $k_i$, and we train a related classifier. 
Then we compute the prediction accuracy as the number of times the 
correct $k_i$ is predicted divided by $n$. 

Table~\ref{tab:acc} presents the obtained accuracies with the computation time on an Intel 
machine 8 VCPU 32GB RAM. 
As a baseline we consider a classifier that always predicts the key value that has the greatest frequency in the database. 


\begin{table}[!h]
\centering
{\setlength\tabcolsep{3pt} % default value: 6pt
\begin{tabular}{|l|l|>{\columncolor{Gray}}l|l|>{\columncolor{Gray}}l|l|>{\columncolor{Gray}}l|l|>{\columncolor{Gray}}l|l|} 
\hline 
\rowcolor{Gray} 
              &         & \multicolumn{2}{c|}{m=1}  & \multicolumn{2}{c|}{m=5}  & \multicolumn{2}{c|}{m=10}  & \multicolumn{2}{c|}{m=20} \\ \hline 
$|\calK|$     &  BL\%   &  Acc.\%   &  Time         &  Acc.\%   &  Time         &  Acc.\%   &  Time          &  Acc.\%   &  Time         \\ \hline  
6             &  19.1   &  100      &  08m 16s      &           &               &           &                &           &               \\ \hline 
11            &  10.7   &  100      &  14m 05s      &           &               &           &                &           &               \\ \hline 
16            &  8.1    &  100      &  21m 03s      &           &               &           &                &           &               \\ \hline 
21            &  6.3    &  99       &  24m 56s      &           &               &           &                &           &               \\ \hline 
26            &  5.1    &  93       &  32m 09s      &  100      & 2h 30m 48s    &  100      & 5h 2m 51s      &           &               \\ \hline 
31            &  4.6    &  78       &  36m 00s      &  99.9     & 2h 58m 27s    &  100      & 6h 2m 43s      &           &               \\ \hline 
41            &  4.3    &  55       &  49m 12s      &  98.2     & 3h 47m 34s    &  100      & 7h 42m 27s     &           &               \\ \hline 
46            &  3.8    &  44.9     &  54m 25s      &  91.5     & 4h 18m 32s    &  100      & 8h 32m 29s     &           &               \\ \hline 
51            &  3.7    &  36.7     &  57m 54s      &  77.9     & 4h 37m 16s    &  99.6     & 9h 15m 52s     &           &               \\ \hline 
56            &  3.7    &  29.8     &  1h 1m 29s    &  64.1     & 5h 5m 4s      &  96.5     & 10h 0m 26s     &           &               \\ \hline 
61            &  3.6    &  22.6     &  1h 4m 8s     &  47.7     & 5h 21m 12s    &  85.5     & 10h 45m 39s    & 98.50     & 21h 29m 18s   \\ \hline
66            &  3.6    &  19.8     &  1h 7m 12s    &  44.0     & 5h 27m 22s    &  83.1     & 10h 50m 28s    & 97.0      & 21h 55m 21s   \\ \hline
71            &  3.6    &  24.9     &  1h 11m 41s   &  38.6     & 6h 1m 19s     &  66.3     & 12h 4m 45s     & 95.70     & 24h 10m 06s   \\ \hline
\end{tabular} 
}
\label{tab:acc}
\caption{Accuracy and computation time.} 
\end{table}


\paragraph{Experiment 2}
We assume that 10 keys are unknown, say, $k_1, \ldots, k_10$. 
For key $k_i$ we train a classifier to predict its value. The training data is generated as before by iterating over all possible lables. 
For other unknown keys, we set their values either to the already predicted ones ($j<i$) or to a random fixed lable ($j>i$).  

We use the privacy breach increase definetion from~\cite{DBLP:conf/icde/CormodePSSY13}: $\beta\% = \frac{Acc.}{BL-1}*100$. 
For Acc. we consider the average accuracy of all unknown records. 


\begin{table}[!h]
\centering
{\setlength\tabcolsep{2pt} % default value: 6pt
\begin{tabular}{|l|l|l|l|l|l|l|l|l|l|l|l|l|l|l|l|} 
\hline 
\rowcolor{Gray} 
$|\calK|$     &  H  &  S      & BL\%   &  Acc.1  &  Acc.2  &  Acc.3  &  Acc.4 &  Acc.5  &  Acc.6  &  Acc.7  &  Acc.8  &  Acc.9 &  Acc.10  & $\beta$\%  & Time          \\ \hline  
11            &  10 &  $10^3$ & 10.7   &  13.10  &  13.30  &  14.30  & 11.80  &  12.00  &  15.30  &  14.30  &  13.80  & 16.10  &  13.90   & 142        &  21h 29m 46s     \\ \hline  %Average Ac. = 13.79 
10            & 100 &  $10^3$ & 10.7   &  13.00  &  14.00  & 14.50   &  15.10  & 15.00  &  14.60  &  13.40  &  15.40  &  15.40  & 16.10   & 157        &  8d 22h 39m 26s   \\ \hline %Average Ac. = 14.65
\end{tabular} 
}
\label{tab:acc}
\caption{Exp 2: Accuracy and computation time. H is number of histograms per label and S is the number of samples (queries) per histogram.} 
\end{table}

\bibliographystyle{abbrv}
\bibliography{biblio} 
\end{document}
